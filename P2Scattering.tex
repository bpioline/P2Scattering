
\documentclass[12pt]{JHEP3}
\usepackage{epsfig,amsfonts,amssymb,amsmath} 
 
%\setlength{\topmargin}{.5cm}
%\setlength{\textheight}{23.1cm}Denef
\setlength{\textwidth}{14.5cm}
%\setlength{\evensidemargin}{2.4cm}
%\setlength{\oddsidemargin}{2.4cm} 
% \renewcommand{\baselinestretch}{1.1}
% 
    \def\CI {{\cal I}}
    \def\CF {{\cal F}}
   
\usepackage{marginnote}
\reversemarginpar
%\newcommand{\newnotea}{ \marginnote{\bf New in v3.1:}[-4mm]}
%\newcommand{\newnoteb}{ \marginnote{\bf New in v2.1:}[-4mm]}
 
 \newcommand{\newnotea}{}
\newcommand{\newnoteb}{}


% from Bourjaily arXiv:1212.6974
%\pdfoutput=1
%\let\ifpdf\relax
\usepackage{color}
\let\normalcolor\relax

\newcommand{\mathematica}[3]{\vspace{0.35cm}\noindent\boxed{\begin{minipage}{#1\textwidth}\begin{tabular}{lp{13cm}}{\color{paper_blue}{\scriptsize{\tt In[1]:}}\raisebox{-0.65pt}{{\scriptsize{\tt=}}}}&{\tt #2}\\{\color{paper_blue}{\scriptsize {\tt Out[1]:}}\raisebox{-0.65pt}{{\scriptsize{\tt=}}}}&{\tt #3}\end{tabular}\end{minipage}}\vspace{0.35cm}}

\definecolor{varcolor}{rgb}{0.1,0.55,0.25}
\definecolor{functioncolor}{rgb}{0.1,0.35,0.75}
\definecolor{paper_blue}{rgb}{0.3,0.2,0.75}
\definecolor{paper_red}{rgb}{0.65,0.1,0.15}
\definecolor{paper_green}{rgb}{0.05,0.35,0.125}
\definecolor{paper_grey}{gray}{0.375}
\definecolor{perm}{rgb}{0.1,0.45,0.85}
\definecolor{deemph}{rgb}{0.7,0.7,0.7}
\setlength{\leftmargini}{21pt}
\newcommand{\vardef}[1]{{\color{varcolor}{\sl #1}\rule[-1.05pt]{7.5pt}{.75pt}}}
\newcommand{\vardefms}[1]{{\color{varcolor}{\sl #1}\rule[-1.05pt]{15pt}{.75pt}}}
\newcommand{\vardefo}[1]{{\color{varcolor}{\sl #1}\rule[-1.05pt]{7.5pt}{.75pt}{\bf{\sl :}}}}
\newcommand{\vardefoo}[1]{{\color{varcolor}{\sl #1}\rule[-1.05pt]{15pt}{.75pt}{\bf{\sl :}}}}
\newcommand{\defn}[3]{~\\[-35pt]\begin{itemize}\item[]\indent\hspace{-21pt}$\bullet$\hspace{-.75pt} {\tt {\color{functioncolor}#1}\![}#2{\tt\,]\!:}#3\end{itemize}\vspace{-7pt}
}
\newcommand{\defnNA}[3]{~\\[-30pt]\begin{itemize}\item[]\indent\hspace{-21pt}$\bullet$\hspace{-.75pt} {\tt {\color{functioncolor}#1}\!}#2{\tt\,\!:}#3\end{itemize}\vspace{-10pt}}
\newcommand{\defntb}[4]{~\\[-30pt]\begin{itemize}\item[]\indent\hspace{-21pt}$\bullet$\hspace{-.75pt} {\tt {\color{functioncolor}#1}\![}#2{\tt\,]\![}#3{\tt\,]\!:}#4\end{itemize}\vspace{-10pt}}
\newcommand{\defvar}[2]{~\\[-30pt]\begin{itemize}\item[]\indent\hspace{-21pt}$\bullet$\hspace{-.75pt} \var{#1}: #2\end{itemize}\vspace{-10pt}}
\newcommand{\var}[1]{{\tt{\color{varcolor}{\tt #1}}}}
\newcommand{\ind}{\hspace{4ex}}
\newcommand{\fun}[1]{{\color{functioncolor}{\tt #1}}}

%%%%%%%%%%%%%%%

\newcommand{\hab}{}



\newcommand{\bZ}{\mathbb{Z}}
\newcommand{\cC}{\mathcal{C}}
\newcommand{\cQ}{\mathcal{Q}}
\newcommand{\cO}{\mathcal{O}}
\newcommand{\cV}{\mathcal{V}}

\newcommand{\IZ}{\mathbb{Z}}
\newcommand{\IP}{\mathbb{P}}
\newcommand{\IH}{\mathbb{H}}
\newcommand{\IC}{\mathbb{C}}
\newcommand{\kk}{k}

\newcommand{\IR}{\mathbb{R}}


\newcommand{\nn}{\nonumber \\}
\newcommand{\be}{\begin{equation}}
\newcommand{\ee}{\end{equation}}
\newcommand{\ben}{\begin{eqnarray}\displaystyle}
\newcommand{\een}{\end{eqnarray}}

\newcommand{\refb}[1]{(\ref{#1})}
\newcommand{\p}{\partial}

\newcommand{\ta}{\tilde\alpha}
\newcommand{\cc}{\tilde c}
\newcommand{\ha}{\hat\alpha}
\newcommand{\dx}{c}

\newcommand\bOm{\bar\Omega}
\newcommand\tOm{\widetilde\Omega}

\newcommand{\cR}{\mathcal{R}}
\newcommand{\cF}{\mathcal{F}}
\newcommand{\cN}{\mathcal{N}}
\newcommand{\cM}{\mathcal{M}}
\newcommand{\gref}{g_{\rm Coulomb}}
\newcommand{\gR}{G_{\rm Higgs}}
\newcommand{\gRa}{g_{\rm Higgs}}
\newcommand{\QR}{Q_{\rm Higgs}}
\newcommand{\bQR}{\bar Q_{\rm Higgs}}
%\newcommand{\QC}{Q_{\rm Coulomb}}
%\newcommand{\bQC}{{\bar Q}_{\rm Coulomb}}

\newcommand{\QC}{\Omega}
\newcommand{\bQC}{\bOm}

\newcommand{\gC}{G_{\rm Coulomb}}
\newcommand{\OmS}{\Omega_{\rm S}} 
\newcommand{\Omatt}{\Omega_*} 
\newcommand{\I}{{\rm i}}
\DeclareMathOperator{\ch}{ch}
%\newcommand{\eqref}{\refb}

\renewcommand{\theequation}{\thesection.\arabic{equation}}

%\newcommand{\href}[1]{#1}

\def\newdefy{\vardef{y}}

\title{P2Scattering.m v1.3}
 
\author{Bruno Le Floch and Boris Pioline}

\abstract{The  {\sc Mathematica} package {\tt P2Scattering.m}, first released alongside our work
with Pierrick Bousseau and Pierre Descombes \cite{}, provides a suite of routines to study
the scattering diagram for the derived category $\cC=D^b(K_{\IP^2})$ of coherent sheaves on 
local $\IP^2$ along the large volume, orbifold and  $\Pi$-stability slices in the space of Bridgeland stability conditions, and extracting the corresponding generalized Donaldson-Thomas invariants $\Omega_Z(\gamma)$. We provide a list of all routines and give a few examples. 
 }

\begin{document}


\maketitle

\baselineskip 15pt

\tableofcontents

\medskip

\section{Summary}

The  {\sc Mathematica} package {\tt P2Scattering.m} provides a suite of routines to study
the scattering diagram for the derived category $\cC=D^b(K_{\IP^2})$ of coherent sheaves on 
local $\IP^2$ along the large volume, orbifold and  $\Pi$-stability slices in the space of Bridgeland stability conditions, and extracting the corresponding generalized Donaldson-Thomas invariants $\Omega_Z(\gamma)$. 

Following \cite{P2Global}, the charge vector $\gamma$ stands either for 
the Chern vector $[r,d,\ch_2]$, its integral cousin $[r,d,\chi)$ where $\chi=r+\frac32d+\ch_2$,
or the dimension vector $(n_1,n_2,n_3)$ associated to the tilting sequence 
\be
(E_1=i_*(\cO), \quad E_2=i_*(\Omega(1))[1], \quad E_3 = i_*(\cO(-1))[2])
\ee
such that
\be
(n_1,n_2,n_3) = (-\frac32 d-\ch_2-r,  -\frac12 d-\ch_2, \frac12 d-\ch_2) = (-\chi, r+d-\chi,r+2d-\chi)
\ee
The central charge is given by $Z(\gamma)=-r T + d T_D -\ch_2$, where $(T,T_D)$ are 
\be
T= s+\I t\ ,\quad T_D = \frac12 (s+\I t)^2 
\ee 
for the large volume slice, or 
\be
\label{Eichler0}
 \begin{pmatrix} T \\ T_D  \end{pmatrix}
= \begin{pmatrix} -\frac12 \\ \frac13  \end{pmatrix} 
+  \int_{\tau_o}^{\tau} \begin{pmatrix} 1 \\u \end{pmatrix} \, 
\, C(u)  du
\ee
for the $\Pi$-stability slice, where $C (\tau) = \frac{\eta(\tau)^9}{\eta(3\tau)^3}$ is a
weight 3 Eisenstein series for $\Gamma_1(3)$ and $\tau_o=-\frac12+\frac{i}{2\sqrt3}$ is 
the orbifold point. The geometric rays in the scattering diagram are then the real codimension-one 
loci 
\be
\cR_\psi(\gamma) = \{ \tau : \Re(e^{\I\psi} Z(\gamma)=0 \}
\ee
For the large volume central charge and $\psi=0$, the scattering diagram reduces to the one
constructed in\cite{Bousseau:2019ift}.
Finally, around the orbifold slice, we 
consider a two-dimensional slice $\theta_1+\theta_2+\theta_3=-\sin\epsilon$ in the
space of King stability parameters $\theta_i=\Im(e^{-\I\epsilon} Z_\tau(\ch E_i))$, and the rays are
the real-codimension one loci $n_1\theta_1+n_2\theta_2+n_3\theta_3=0$. 




\medskip

The package file {\tt P2Scattering.m} and various example files 
can be obtained from  {\tt https://github.com/bpioline/P2Scattering}

\subsection{Basic usage}

Assuming that the file {\tt P2Scattering.m} is present in the user's {\sc Mathematica} Application 
directory, the package is loaded by entering 

\mathematica{1.0}{ <<P2Scattering`}{P2Scattering 1.2 -- A package for evaluating DT invariants on $K_{P^2}$}

If the file  {\tt P2Scattering.m} has not yet been copied in the user's {\sc Mathematica} Application 
directory but is in the same directory as the notebook, evaluate instead

\mathematica{1.0}{SetDirectory[NotebookDirectory[]]; <<P2Scattering`}{P2Scattering 1.2- A package for evaluating evaluating DT invariants on $K_{P^2}$}

For given charge $\gamma=[r,d,\chi)$ and point $(s,t)$ on the large volume slice, the trees contributing to the index $\Omega_{(s,t)}(\gamma)$ can be found by using the routine $\fun{ScanAllTrees}$, 
for example for $\gamma=[3,0,0)$ through the point $(s,t)=(-\frac32,2)$,

\mathematica{1.0}{LiTrees=ScanAllTrees[\{0, 3, 0\}, \{-3/2, 2\}]
}{
\{\{-3 Ch[-2], 3 Ch[-1]\}, \{-Ch[-3], Ch[0]\}\}
}

\mathematica{1.0}{ScattDiagLV[LiTrees, 0]
}{
\includegraphics[width=6cm]{figs/Scatt30.pdf}
}

\mathematica{1.0}{Limit[EvaluateKronecker[ScattIndex[LiTrees]], y -> 1]
}{
\{18,9\}
}
reproducing the GV invariant $N_3^0=27$. Similarly, one can find the trees contributing near
the orbifold point using \fun{McKayScanAllTrees}: for the same charge, corresponding to dimension vector $(0,3,6)$, a single tree contributes in the anti-attractor chamber, with index 18,

\mathematica{1.0}{
LiTrees = McKayScanAllTrees[chiton[\{0, 3, 0\}]]; 
LiTrees /. McKayrep
}{
$\{\{ 3\gamma_2,6 \gamma_3\}\}$
}

\mathematica{1.0}{
Limit[EvaluateKronecker[McKayScattIndex[LiTrees]], y ->1]
}{
\{18\}
}

\mathematica{1.0}{
Show[McKayInitialRays[2], McKayScattDiag[LiTrees]]
}{
\includegraphics[width=6cm]{figs/McKayScattering036.pdf}
}

More  examples can be found on the GitHub repository.


\subsection{Online documentation}

The package contains many more routines, documented below, which can be used independently. Basic inline documentation can be obtained by typing e.g. 

\mathematica{1.02}{?EichlerT}{EichlerT[tau\_] gives numerical value of T[tau] using
Eichler integral, by mapping back to fundamental domain $F_C$}
   
\subsection{History}   
   
The first version of this package was released together with the preprint \cite{P2Global}. 


\section{Symbols and global variables}

\defvar{tau}{K\"ahler modulus in Poincar\'e upper half-plane}

\defvar{tau1}{Real part of \var{tau}}

\defvar{tau2}{Imaginay part of \var{tau}}

\defvar{taup}{Fricke transform $\tau'=-1/(3\tau)$}

 \defvar{y}{refinement parameter}
 
 \defvar{a}{Parameter running from 0 to 1 along rays}

\defn{Ch}{\vardef{m}}{denotes the charge vector for $\cO(m)$ if $m$ is integer; 
More generally, if $m$ is rational, it denotes the charge of the image of $\cO$
which is massless at $\tau=m$}

\defn{Ch}{\vardef{m}][\vardef{n}}{denotes the charge vector for $\cO(m)[n]$}


\defn{Kr$_m$}{\vardef{p},\vardef{q}}{denotes the index of the Kronecker quiver with m arrows, dimension vector $(p,q)$}
 
\defvar{gam1}{Charge vector $[-1,0,-1)$ for $E_1[-1]=\cO(0)[-1]$}

\defvar{gam2}{Charge vector $[2,-1,0)$ for $E_2[-1]=\Omega(1)[0]$}

\defvar{gam3}{Charge vector $[-1,1,0)$ for $E_3[-1]=\cO(-1)[1]$}

\defvar{tauO}{orbifold point $-1/2+i/(2\sqrt{3})$}

\defvar{QuantumVolume}{Quantum period $\cV= \Im T(0) \simeq 0.462758$}

\defvar{MLV}{Monodromy matrix in $[r,d,\chi)$ basis around $\tau=i\infty$}
\defvar{MCon}{Monodromy matrix  in $[r,d,\chi)$ basis around at $\tau=0$}
\defvar{MOrb}{Monodromy matrix  in $[r,d,\chi)$ basis around orbifold point $\tau_o$}
\defvar{MOrbp}{Monodromy matrix  in $[r,d,\chi)$ basis around orbifold point $\tau_o+1$}


\defn{Trees}{\vardef{r},\vardef{d},\vardef{chi}}{gives the list of precomputed trees at large volume, when available}
\defn{McKayTrees}{\{\vardef{n1},\vardef{n2},\vardef{n3}\}}{gives the list of precomputed  trees around the orbifold point, when available}
\defvar{ExcepSlopes}{List of slopes of exceptional bundles between -3 and 4}
\defvar{ListSubsetsAny}{Precomputed list of all binary splits of Range[n] for n=2..10, used by \fun{ListStableTrees}}
\defvar{FourierC}{List of the first 50 Fourier coefficients of Eisenstein series $C(\tau)$}
\defvar{FourierCp}{List of the first 50 Fourier coefficients of Eisenstein series $C'(\tau')$}

%\defvar{Initialtau1}{Initial value of tau1 for FindRoot}
%\defvar{Initialtau2}{Initial value of tau2 for FindRoot}



\section{Operations on Chern vectors and dimension vectors}

\defn{Euler}{\{\vardef{r},\vardef{d},\vardef{chi}\},\{\vardef{rr},\vardef{dd},\vardef{cchi}\}}
{computes the Euler form on $D^b(\IP^2)$}
\defn{DSZ}{\{\vardef{r},\vardef{d},\vardef{chi}\},\{\vardef{rr},\vardef{dd},\vardef{cchi}\}}
{computes the antisymmetrized Euler form $3(r d'-r' d)$ on $D^b(\IP^2)$}
\defn{McKayDSZ}{\{\vardef{n1},\vardef{n2},\vardef{n3}\},\{\vardef{nn1},\vardef{nn2},\vardef{nn3}\}]}
{computes the antisymmetrized quiver Euler form}
\defn{Disc}{\{\vardef{r},\vardef{d},\vardef{chi}\}}
{computes the discriminant $\Delta(\gamma)=d^2-2r(\chi-r-\frac32 d))/(2r^2)$}
\defn{Disch2}{\{\vardef{r},\vardef{d},\vardef{ch2}\}}
{computes the discriminant $\Delta(\gamma)=(d^2-2r \ch_2)/(2r^2)$}
\defn{DiscR}{\{\vardef{r},\vardef{d},\vardef{chi}\}}
{computes the rescaled discriminant $r^2 \Delta(\gamma) = d^2-2r(\chi-r-\frac32 d)$}
\defn{GenSlope}{\{\vardef{r},\vardef{d},\vardef{chi}\}}
{computes the slope $d/r$ if $r\neq 0$, or $\chi/d-\frac32$ if $r=0$}
\defn{DimGieseker}{\{\vardef{r},\vardef{d},\vardef{chi}\}}
{computes expected dimension of moduli space of Gieseker-semistable sheaves}
\defn{DimMcKay}{\{\vardef{n1},\vardef{n2},\vardef{n3}\}}
{computes the dimension of quiver moduli space in anti-attractor chamber}
\defn{ch2tochi}{\vardef{r},\vardef{d},\vardef{ch2}}
{produces corresponding $[r,d,\chi)$}
\defn{chitoch2}{\vardef{r},\vardef{d},\vardef{chi}}
{produces corresponding $[r,d,\ch_2]$}
\defn{chiton}{\{\vardef{r},\vardef{d},\vardef{chi}\}}
{produces the corresponding dimension vector $(n_1,n_2,n_3)$}
\defn{ntochi}{\{\vardef{n1},\vardef{n2},\vardef{n3}\}]}
{produces the corresponding charge vector $[r,d,chi)$}
\defn{SpecFlow}{\{\vardef{r},\vardef{d},\vardef{ch2}\},\vardef{k}}
{computes the translated charge vector $[r,d,\chi)(k)$}
\defn{SpecFlowch2}{\{\vardef{r},\vardef{d},\vardef{ch2}\},\vardef{k}}
{computes the translated charge vector $[r,d,\ch_2](k)$}
\defn{repCh}{}
{Replacement rule mapping $\fun{Ch}[s]$ or $\fun{Ch}[s][n]$ to their charge vector $[r,d,\chi)$}
\defn{repCh2}{}
{Replacement rule mapping $\fun{Ch}[s]$ or $\fun{Ch}[s][n]$ to their charge vector $[r,d,\ch_2]$}
\defn{repChn}{}
{Replacement rule mapping $Ch[m]$ into \fun{chiton}$[\{1,m,1+m(m+3)/2\}$]}
\defn{repChO}{}
{Replacement rule mapping $Ch[m_]$ to string $O(m)$}
\defn{repKr}{}
{replaces $\var{Kr}_m[p,q]$ by $1$}
\defn{LPCurve}{\vardef{mu}}
{computes the Drezet-Le Potier curve $\delta(\mu)$}
\defn{ExcepToChi}{\vardef{mu}}
{gives the Chern vector [$r,d,\chi)$ of the exceptional bundle of slope $\mu$}
\defn{CPointchi}{\vardef{tau}}
{gives the charge vector $[r,d,\chi)$ of an object that becomes massless at $\tau$ (assuming
$\tau$ is a rational number)}
\defn{CPointch2}{\vardef{tau}}
{gives the charge vector $[r,d,\ch_2]$ of an object that becomes massless at $\tau$ (assuming
$\tau$ is a rational number)}
\defn{EvaluateKronecker}{\vardef{f}}
{replaces each $Kr_m[p,q]$ with the index of the Kronecker quiver with $m$ arrows and dimension vector $(p,q)$, using routines taken from \fun{CoulombHiggs.m} package}
\defn{McKayrep}{}
{replaces \{\vardef{n1},\vardef{n2},\vardef{n3}\} by $n_1 \gamma_1+n_2  \gamma_1+n_3\gamma_3$}




\section{K\"ahler moduli space and central charge for $\Pi$-stability}

\defn{ToFundDomainO}{\vardef{tau}}
{produces $\{\tau',M'\}$ such that $M'\cdot\tau'=\tau$ and $\tau'$ lies in fundamendal domain of $\Gamma_1(3)$ centered around orbifold}
\defn{ToFundDomainC}{\vardef{tau}}
{ produces $\{\tau',M'\}$ such that $M'\cdot\tau'=\tau$ and $\tau'$ lies in fundamendal domain of $\Gamma_1(3)$ centered around conifold}
\defn{ToFundDomainCSeq}{\vardef{tau}}
{produces $\{\tau',M'\}$ such that $M'\cdot\tau'=\tau$ and $\tau'$ lies in fundamendal domain of $\Gamma_1(3)$ centered around conifold, M is a string of U,V,T[m] generators}
\defn{ToFundDomainOApprox}{\vardef{tau},\vardef{precision}}
{produces $\{\tau',M'\}$ such that $M'\cdot\tau'=\tau$ and $\tau'$ lies in fundamendal domain of $\Gamma_1(3)$ centered around orbifold}
\defn{ToFundDomainCApprox}{\vardef{tau},\vardef{precision}}
{produces $\{\tau',M'\}$ such that $M'\cdot\tau'=\tau$ and $\tau'$ lies in fundamendal domain of $\Gamma_1(3)$ centered around conifold}
\defn{ApplyGamma13Lift}{\vardef{M},\vardef{tau}}
{produces the image of $\tau$ under the 3x3 monodromy matrix $M$}
\defn{MonodromyOnCharge}{\vardef{M},\{\vardef{r},\vardef{d},\vardef{chi}\}}
{computes the image of charge vector $[r,d,\chi)$ under sequence of monodromies}
\defn{MonodromyOnTau}{\vardef{M},\vardef{tau}}
{computes the image of $\tau$ under sequence of monodromies}
\defn{FundDomainO}{\vardef{k}}
{produces the Graphics directives for the fundamental domain of $\Gamma_1(3)$ on the interval 
$[-\frac12+k,\frac12+k]$}
\defn{FundDomainC}{\vardef{k}}
{produces the Graphics directives for the fundamental domain of $\Gamma_1(3)$ on the interval $[k,k+1]$}
\defn{FundDomain3}{\vardef{k}}
{produces the Graphics directives for the fundamental domain of $\Gamma_1(3)+\IZ_3$ images on the interval $[k,k+1]$}
\defn{FundDomainRulesC}{\vardef{k}}
{gives a list of rules $\tau\to \tau(a)$ for boundaries of \fun{FundDomainC[k]} parametrized by $0<a<1$}
\defn{FundDomainRulesO}{\vardef{k}}
{ gives a list of rules $\tau\to \tau(a)$ for boundaries of  \fun{FundDomainO[k]} parametrized by $0<a<1$}
\defn{FundDomainRules3}{\vardef{k}}
{gives a list of rules $\tau\to \tau(a)$ for boundaries of \fun{FundDomain3[k]} parametrized by $0<a<1$}
\defn{EichlerC}{\vardef{tau}}
{numerically evaluates the Eisenstein series $C(\tau)$}
\defn{EichlerCp}{\vardef{taup}}
{numerically evaluates the Eisenstein series $C'(\tau')$}
\defn{Eichlerf1}{\vardef{tau}}{
evaluates $2\pi i (\tau+1/2)+\bar f_1(\tau)$}
\defn{Eichlerf2}{\vardef{tau}}
{evaluates $\frac12 (2\pi i)^2(\tau+\frac12)^2+2\pi i(\tau+\frac12)\bar f_1(\tau)+\bar f_2(\tau)$}
\defn{Eichlerf1b}{\vardef{tau}}
{$=\sum_{n\geq 1} \frac{c_n}{n} e^{2\pi i n \tau}$}
\defn{Eichlerf2b}{\vardef{tau}}
{$=-\sum_{n\geq 1} \frac{c_n}{n^2} e^{2\pi i n \tau}$}
\defn{Eichlerf1p}{\vardef{taup}}
{$=\sum_{n\geq 1} \frac{c_n}{n} e^{2\pi i n \tau}$}
\defn{Eichlerf2p}{\vardef{taup}}
{$=2\pi i \tau' f_1'(\tau')-\sum_{n\geq 1} \frac{c_n}{n^2} e^{2\pi i n \tau}$}
\defn{EichlerTLV}{\vardef{tau}}
{gives numerical value of $T(\tau)$ using Eichler integral near LV point}
\defn{EichlerTDLV}{\vardef{tau}}
{gives numerical value of $T_D(\tau)$ using Eichler integral near LV point}
\defn{EichlerTC}{\vardef{tau}}
{gives numerical value of $T(\tau)$ using Eichler integral near conifold point}
\defn{EichlerTDC}{\vardef{tau}}
{gives numerical value of $T_D(\tau)$ using Eichler integral near conifold point}
\defn{EichlerZ}{\{\vardef{r},\vardef{d},\vardef{chi}\},\vardef{tau}}
{gives numerical value of $Z_\tau(\gamma)$ by mapping $\tau$ back to fundamental domain $\cF_C$}
\defn{EichlerZch2}{\{\vardef{r},\vardef{d},\vardef{ch2}\},\vardef{tau}}
{gives numerical value of $Z_\tau(\gamma)$ by mapping $\tau$ back to fundamental domain $\cF_C$}
\defn{EichlerZch2LV}{\{\vardef{r},\vardef{d},\vardef{ch2}\},\vardef{tau}}
{gives numerical value of $Z_\tau(\gamma)$ using Eichler integral at large volume}
\defn{EichlerT}{\vardef{tau}}
{gives numerical value of $T(\tau)$ using Eichler integral, by mapping back to fundamental domain $\cF_C$}
\defn{EichlerTD}{\vardef{tau}}
{gives numerical value of $T_D(\tau)$ using Eichler integral, by mapping back to fundamental domain $\cF_C$}
\defn{EichlerTtilde}{\vardef{tau}}
{gives numerical value of $\tilde{T}=\frac{1}{2\sqrt3}(T-\frac12)$ using Eichler integral, by mapping back to fundamental domain $\cF_C$}
\defn{EichlerTDtilde}{\vardef{tau}}
{ gives numerical value of $\tilde{T}_D=T_D-\frac12 T-\frac{1}{12}$ using Eichler integral, by mapping back to fundamental domain $\cF_C$}



\section{Large volume scattering diagram}

\defn{ZLV}{\{\vardef{r},\vardef{d},\vardef{chi}\},\{\vardef{s},\vardef{t}\}}
{computes the large volume central charge $-\frac12 r (s+i t)^2+d (s+i t)-(r-\frac32d-\chi)$}
\defn{ZLVch2}{\{\vardef{r},\vardef{d},\vardef{ch2}\},\{\vardef{s},\vardef{t}\}}
{computes the central charge $-\frac12 r (s+i t)^2+d (s+i t)-\ch_2$}
\defn{Wall}{\{\vardef{r},\vardef{d},\vardef{chi}\},\{\vardef{rr},\vardef{dd},\vardef{cchi}\},\{\vardef{s},\vardef{t}\}}
{computes $\Im[Z(\gamma)\overline{Z(\gamma')}]$}
\defn{Wallch2}{\{\vardef{r},\vardef{d},\vardef{ch2}\},\{\vardef{rr},\vardef{dd},\vardef{cch2}\},\{\vardef{s},\vardef{t}\}}
{computes $\Im[Z(\gamma)\overline{Z(\gamma')}]$}
\defn{TreeCharge}{\vardef{tree}}
{computes the total charge $[r,d,\chi)$ carried by tree (or list of trees).}
\defn{TreeChargech2}{\vardef{tree}}
{computes  the total charge $[r,d,\ch_2]$ carried by tree (or list of trees).}
\defn{TreeTop}{\vardef{tree}}
{ computes the (s,t) coordinate of the root of the tree}
\defn{TreeConstituents}{\vardef{tree}}
{gives the flattened list of constituents of Tree}
\defn{FlipTree}{\vardef{tree}}
{ constructs the reflected tree under $Ch[m]\to -Ch[-m]$}
\defn{ShiftTree}{\vardef{tree},\vardef{k}}
{ constructs the shifted tree under $Ch[m]\to Ch-m+k]$}
\defn{ScattCheck}{\vardef{tree}}
{returns $\{charge,\{x,y\}\}$ of the root vertex if \var{Tree} is consistent, otherwise 
$\{total charge,\{\}\}$}
\defn{ScattSort}{\vardef{Litree}}
{ sorts trees by growing radius of wall}
\defn{ScattGraph}{\vardef{tree}}
{extracts the list of vertices and adjacency matrix of Tree}
\defn{xytost}{\{\vardef{x},\vardef{y}\}}
{computes $\{x,\sqrt{x^2+2y}\}$}
\defn{sttoxy}{\{\vardef{s},\vardef{t}\}}
{computes $\{s,-\frac12(s^2-t^2)\}$}
\defn{IntersectRays}{\{\vardef{r},\vardef{d},\vardef{chi}\},\{\vardef{rr},\vardef{dd},\vardef{cchi}\},\vardef{z},\vardef{zz}}
{returns intersection point $(x,y)$ of two rays if the intersection point lies upward from $z$ and $z'$, or
the empty set otherwise; the arguments $z,z'$ can be omitted.}  
\defn{IntersectRaysSt}{\{\vardef{r},\vardef{d},\vardef{chi}\},\{\vardef{rr},\vardef{dd},\vardef{cchi}\},\vardef{psi}}
{ returns intersection point $(s,t)$ for scattering rays with phase $\psi$, or the empty set if they are collinear}
\defn{TestBranch}{\{\vardef{r},\vardef{d},\vardef{chi}\},\vardef{s}}
{returns \var{True} if $(s,\cdot)$ is on the branch with $\Im Z(\gamma)>0$, \var{False} otherwise}
\defn{Rayt}{\{\vardef{r},\vardef{d},\vardef{chi}\},\vardef{s},\vardef{psi}}
{ computes  $t(s)$ for the ray $\Re[e^{-i\psi} Z^{\rm LV}(\gamma)]=0$} 
\defn{Rays}{\{\vardef{r},\vardef{d},\vardef{chi}\},\vardef{t},\vardef{psi}}
{ computes  $s(t)$ for the ray $\Re[e^{-i\psi} Z^{\rm LV}(\gamma)]=0$} 
\defn{Raytch2}{\{\vardef{r},\vardef{d},\vardef{chi}\},\vardef{s},\vardef{psi}}
{ computes  $t(s)$ for the ray $\Re[e^{-i\psi} Z^{\rm LV}(\gamma)]=0$} 
\defn{Raysch2}{\{\vardef{r},\vardef{d},\vardef{chi}\},\vardef{t},\vardef{psi}}
{ computes  $s(t)$ for the ray $\Re[e^{-i\psi} Z^{\rm LV}(\gamma)]=0$} 
\defn{ListStableTrees}{\vardef{LiCh},\{\vardef{s0},\vardef{t0}\}}
{constructs consistent trees from constituents in \var{LiCh} of the form $k_i Ch[m_i]$, 
which are stable at $(s_0,t_0)$}
\defn{ListStableTreesPerturb}{\vardef{LiCh},\{\vardef{s0},\vardef{t0}\}}
{constructs consistent trees from constituents in \var{LiCh} of the form $k_i Ch[m_i]$, 
which are stable at $(s_0,t_0)$ after perturbing  the $m_i$'s}
\defn{ListStablePlanarTrees}{\vardef{LiCh},\{\vardef{s0},\vardef{t0}\}}
{constructs consistent planar trees from constituents in LiCh of the form $k_i Ch[m_i]$, 
which are stable at $(s_0,t_0)$}
\defn{ScanConstituents}{\{\vardef{r},\vardef{d},\vardef{chi}\},\{\vardef{mmin},\vardef{mmax}\},
\{\vardef{nmin},\vardef{nmax}\},\vardef{phimax}}
{ searches possible list of constituents  with slope in $[mmin,mmax]$, number of constituents in $[nmin,nmax]$, electric potential less than phimax and charges adding up to $[r,d,\chi)$}

\defn{ScanAllTrees}{\{\vardef{r},\vardef{d},\vardef{chi}\},\{\vardef{s},\vardef{t}\}}
{ constructs all possible trees with charges adding up to $[r,d,\chi)$ leading to an
outgoing ray through the point $(s,t)$; uses
\fun{ScanConstituents} with the most conservative values 
$mmin=s-t, mmax=s+t$, $nmin=1, nmax=2\varphi_{s}(\gamma)$
and $phimax=\varphi_{s}(\gamma)$ }

\defn{ScanBinarySplits}{\{\vardef{r},\vardef{d},\vardef{chi}\},\{\vardef{s0},\vardef{t0}\}}
{produces list of $[r',d',\chi')$ such that $[r,d,\chi)$ can split into $[r',d',\chi')$ along the ray starting at $(s_0,t_0)$}


\defn{ScanKroneckerSplits}{\{\vardef{r},\vardef{d},\vardef{chi}\}}
{scans all possible pairs $\{-k' Ch(m'), k Ch(m)\}$ such that each pair scatters into $[r,d,\chi)$}
\defn{ScattIndex}{\vardef{TreeList}}
{computes the index for each tree in \var{TreeList}; do not trust the result if internal lines have non-primitive charges !}
\defn{ScattIndexInternal}{\vardef{TreeList}}
{computes total charge, list of Kronecker indices associated to each vertex in \var{Tree}}

\defn{ScattDiag}{\vardef{TreeList}}
{ draws scattering diagram in $(x,y)$ plane for each tree in \var{TreeList}} 
\defn{ScattDiagOverlay}{\vardef{TreeList}}
{ overlays scattering diagrams in $(x,y)$ plane for each tree in \var{TreeList}} 
\defn{ScattDiagxy}{\vardef{TreeList},\vardef{psi}}
{ draws scattering diagrams in $(x,y)$ plane overlaying each tree in \var{TreeList}, marking the
 initial points} 
 
 \defn{ScattDiagInternal}{\vardef{Tree}}
{ constructs a list \{total charge, coordinate of root, list of line segments, \{min(x), max(x)\}\}
for large volume scattering 
 in (x,y) coordinates; used by \fun{ScattDiag}} 


\defn{ScattDiagSt}{\vardef{TreeList},\vardef{psi}}
{ draws scattering diagram in $(s,t)$ plane for each trees in \var{TreeList}, approximating
each edge by a straight line and overlaying
the various trees; if $\psi$ is omitted, it is assumed to vanish. }

 \defn{ScattDiagInternalSt}{\vardef{Tree},\vardef{psi}}
{ constructs a list \{total charge, coordinate of root, list of line segments, \{min(s), max(s)\}\}
for large volume scattering 
 in (s,t) coordinates, , approximating
each edge by a straight line;  if $\psi$ is omitted, it is assumed to vanish;
 used by \fun{ScattDiagSt}}  
 
\defn{ScattDiagLV}{\vardef{TreeList},\vardef{psi}}
{draws scattering diagram in $(s,t)$  in $(s,t)$ plane for each trees in \var{TreeList}, using exact hyperbolaes for rays and overlaying
the various trees}

\defn{ScattDiagInternalLV}{\vardef{Tree},\vardef{psi},\vardef{Style}}
{ constructs total charge, coordinate of root and list of line segments in $(s,t)$ coordinates,
$ \{\min(x), \max(x)\}$, using PlotStyle$\to$Style for plotting rays ; used by \fun{ScattDiagLV}} 



\defn{ScattDiagLZ}{\vardef{TreeList}}
{ draws scattering diagram in Li-Zhao $(s,q)$ plane for each trees in \var{TreeList}}


\defn{ScattPhi}{\vardef{TreeList}}
{ overlays scattering diagrams in $(s,\varphi)$ plane for each tree in \var{TreeList}} 

\defn{ScattPhiInternal}{\vardef{Tree}}
{constructs a list \{total charge, coordinate of root, list of line segments, \{min(s), max(s)\}\} in $(s,\varphi)$ coordinates;  used by \fun{ScattPhi}}


\defn{PlotWallRay}{\{\vardef{r},\vardef{d},\vardef{chi}\},\{\vardef{rr},\vardef{dd},\vardef{cchi}\},
\vardef{psi},\{\vardef{L1},\vardef{L2},\vardef{H}\}}
{plots the local scattering $[r,d,\chi)\to [r',d',\chi')+[r-r',d-d',\chi-\chi')$ for phase $\psi$
 in range $L_1<s<L_2, 0<t<H$ }
 
\defn{WallCircle}{[\{\vardef{r},\vardef{d},\vardef{chi}\},\{\vardef{rr},\vardef{dd},\vardef{cchi}\}}{ constructs the graphics directive for the wall of marginal stability in $(s,t)$ coordinates}

\defn{WallLine}{[\{\vardef{r},\vardef{d},\vardef{chi}\},\{\vardef{rr},\vardef{dd},\vardef{cchi}\}}{ constructs the graphics directive for the wall of marginal stability in $(s,q)$ coordinates}

\defn{TreeHue}{\vardef{i},\vardef{n}}
{ specifies the color for the $i$-th tree among a list of $n$ - can be modified at will}

\defn{InitialLabelPosition}{\vardef{m}}
{returns a position $(s,t)$ for the label for an initial ray with slope 
$m$; this position is lowered vertically on each call, using variables \var{LiSlopes} and 
\var{LiHeights} to keep track of earlier calls}

\section{Orbifold scattering diagram}

\defn{McKayRayEq}{\{\vardef{n1},\vardef{n2},\vardef{n3}\},\{\vardef{u},\vardef{v}\}}
{gives the linear form vanishing on the scattering ray}
\defn{McKayVec}{McKayVec[\{\vardef{n1},\vardef{n2},\vardef{n3}\}}
{computes the positive vector along the ray}
\defn{McKayRay}{\{\vardef{n1},\vardef{n2},\vardef{n3}\},\{\vardef{u},\vardef{v}\},\{vardef{k1},\vardef{k2}\},\vardef{tx}}
{produces an arrow from $(u,v)+k_1 w$ to $(u,v)+k_2 w$, where $w$ is the positive vector along the ray, decorated with text \var{tx} at the target}
\defn{McKayScattIndex}{\vardef{TreeList}}
{computes the index for each tree in TreeList; do not trust the result if internal lines have non-primitive charges !}
\defn{McKayScattIndexInternal}{\vardef{Tree}}
{computes total charge, list of Kronecker indices associated to each vertex in Tree}
\defn{McKayListAllConsistentTrees}{\{\vardef{n1},\vardef{n2},\vardef{n3}\}}
{generates consistent scattering trees with leaves carrying charge $\{p,0,0\}, \{0,p,0\},\{0,0,p\}$ adding up to $(n_1,n_2,n_3)$, with non-zero DSZ pairing at each vertex, with distinct support}
\defn{McKayListAllTrees}{\{\vardef{n1},\vardef{n2},\vardef{n3}\}}
{generates all trees with leaves carrying charge $\{p,0,0\}, \{0,p,0\},\{0,0,p\}$ adding up to 
$(n_1,n_2,n_3)$and with non-zero DSZ pairing at each vertex}
\defn{McKayScattCheck}{\vardef{Tree}}
{returns $\{charge,\{u,v\}\}$ of the root vertex if Tree is consistent, otherwise $\{total charge,\{\}\}$}
\defn{McKayScattGraph}{\vardef{Tree}}
{extracts the list of vertices and adjacency matrix of Tree}
\defn{McKayScattDiag}{\vardef{TreeList}}
{draws McKay scattering diagram in (u,v) plane for each tree in \var{TreeList}} 
\defn{McKayScattDiagInternal}{\vardef{Tree}}
{constructs total charge, coordinate of root and list of line segments in $(u,v)$ coordinates; used by \fun{McKayScattDiag}} 
\defn{McKayIntersectRays}{\{\vardef{n1},\vardef{n2},\vardef{n3}\},\{\vardef{nn1},\vardef{nn2},\vardef{nn3}\}}
{returnsthe  intersection point $(u,v)$ of two rays, or empty set if they are collinear}

\defn{McKayIntersectRays}{\{\vardef{n1},\vardef{n2},\vardef{n3}\},\{\vardef{nn1},\vardef{nn2},\vardef{nn3}\},\vardef{z},\vardef{zz}}
{returns intersection point $(u,v)$ of two rays if the intersection point lies upward from $z$ and $z'$, or 
empty set otherwise}
\defn{McKayInitialRays}{\vardef{psi},\vardef{L}}
{draws the initial rays in $(u',v')$ plane, rescaling each arrow by a factor of $L$. If the argument $\psi$
is omitted, it is assumed to be $\frac{\pi}{2}$.}

\section{Exact scattering diagram}
\defn{CriticalPsi}{\vardef{mu}}
{$=\arctan(\mu/\cV)$}


\defn{IntersectExactRaysLV}{\{\vardef{r},\vardef{d},\vardef{chi}\},\{\vardef{rr},\vardef{dd},\vardef{cchi}\},\vardef{psi}}
{returns value of $\tau$ at intersection point of two exact rays using \fun{EichlerTLV} to evaluate the periods, or 0 if they are collinear}
\defn{IntersectExactRaysC}{\{\vardef{r},\vardef{d},\vardef{chi}\},\{\vardef{rr},\vardef{dd},\vardef{cchi}\},\vardef{psi}}
{  returns value of $\tau$ atf intersection point of two exact rays using \fun{EichlerTC} to evaluate the periods, or 0 if they are collinear}
\defn{XY}{\vardef{tau},\vardef{psi}}
{computes the affine coordinates $(x,y)$ such that scattering  rays are straight lines $ry+dx-\ch_2=0$}
\defn{CPointxy}{\vardef{tau}}
{computes the $(x,y)$ coordinate of initial point $Ch[\tau]$  (assuming
$\tau$ is a rational number)}
\defn{IntegralCurve}{\vardef{tauinit},\vardef{tangent},\{\vardef{ainit},\vardef{amin},\vardef{amax}\},\vardef{boundaries}}
{produces a function $f:a\in [0,1]\to \IH$ with $f(ainit)=tauinit$ following the tangent direction (an expression in $\tau$) and stopping at the boundaries (by default: $\{Im\tau =0.01\}$).  The range of integration parameters $\{amin,amax\}$ can be infinite provided the actual rays remain finite by hitting the boundaries.}
\defn{NormalizeFunctionDomain}{\vardef{fun}} 
{rescales the argument of the \fun{InterpolatingFunction} \var{fun} to interval $[0,1]$}
\defn{DtauT}{\vardef{tau}}
{numerically evaluates $\partial_\tau T(\tau)$}
\defn{DtauZch2}{\{\vardef{r},\vardef{d},\vardef{ch2}\}m\vardef{tau}}
{numerically evaluates $\partial_\tau  Z_\tau(\gamma)$}
\defn{DtauZ}{\{\vardef{r},\vardef{d},\vardef{chi}\},\vardef{tau}}
{numerically evaluates $\partial_\tau  Z_\tau(\gamma)$}
\defn{ArgDtauT}{vardef{tau}}
{computes the argument of $T'(\tau)$, between $-\pi$ and $\pi$}
\defn{ArgDtauTD}{vardef{tau}}
{computes the argument of $T'_D(\tau)$, between $-\pi$ and $\pi$}
\defn{ArgDtauZch2}{\{\vardef{r},\vardef{d},\vardef{ch2}\},\vardef{tau}}
{computes the argument of $\partial_\tau  Z_\tau(\gamma)$,  between $-\pi$ and $\pi$}
\defn{UnitDtauT}{vardef{tau}}
{numerically evaluates $\partial_\tau T(\tau)/|\partial_\tau T(\tau)|$}
\defn{UnitDtauDT}{vardef{tau}}
{numerically evaluates $\partial_\tau T_D(\tau)/|\partial_\tau T_D(\tau)|$}
\defn{UnitDtauZ}{\{\vardef{r},\vardef{d},\vardef{chi}\}\vardef{tau}}
{numerically evaluates $\partial_\tau  Z_\tau(\gamma)/|\partial_\tau  Z_\tau(\gamma)|$}
\defn{UnitDtauZch2}{\{\vardef{r},\vardef{d},\vardef{ch2}\}\vardef{tau}}
{numerically evaluates $\partial_\tau  Z_\tau(\gamma)/|\partial_\tau  Z_\tau(\gamma)|$}
\defn{NormalizeApprox}{\vardef{z},\vardef{eps}}
{normalizes $z\in \IC$ to roughly unit length for large z, but behaves smoothly near zero.}
\defn{TotalChargech2}{\vardef{Tree}}
{ gives the total charge vector $[r,d,\ch_2]$ of a given tree (nested list).}
\defn{TotalChargechi}{\vardef{Tree}}
{ gives the total charge vector $[r,d,\chi)$ of a given tree (nested list).}
\defn{ConifoldRay}{\vardef{init},\vardef{psi},\vardef{homshift}}
{gives a function $f:a\in [0,1]\to \IH$ parametrizing the ray starting at the rational number 
init$=\frac{p}{q}$ (with $q\neq 0 \mod 3$}
\defn{RayCh}{\vardef{psi}}
{gives an initial ray starting at 0, namely a function $f:[0,1]\to \IH$ starting (close to) 0 and following a ray where $Z_\tau([1,0,0])=-T_D$ has phase $\psi+\frac{\pi}{2} \mod \pi$.  Shifting $\psi$ by $2\pi$  gives a different ray, corresponding to a homological shift by 2.  Values are cached.}
\defn{RayGeneralch2}{\vardef{psi},\vardef{tauexpr},\vardef{start}}
{gives a function $a\in[0,1]\to\IH$  parametrizing the ray where $Z_\tau(\gamma)$ has phase  
$\psi+\frac{\pi}{2} \mod \pi$.  
The starting point is obtained using \fun{FindRoot}[$\dots, Z_{tauexpr}(\gamma),\dots,start$], see documentation of \fun{FindRoot}.
}
\defn{RayFromInfinity}{\{\vardef{r},\vardef{d},\vardef{chi}\},\vardef{psi}}
{gives a function $f:a\in [0,1] \to \IH$ parametrizing the ray of phase $\psi$ 
starting from the large volume limit}
\defn{StabilityWall}{\vardef{Tree},\vardef{tauinit},\vardef{tangent},\{\vardef{ainit},\vardef{amin},\vardef{amax}\}}
{gives a a function $f:a\in [0,1] \to \IH$ parametrizing the stability wall for the last fusion of the tree.  The tree can also be a pair of charges.  The tauinit is used as a starting point of \fun{FindRoot} along a vertical line.  The last argument can be omitted and defaults to $\{-2,2\}$; it is an interval around the starting point 0, and can be used to restrict the stability wall to only one side of tauinit.}
\defn{TreeToRays}{\vardef{Tree},\vardef{psi}}
{gives the (flat) list of rays (functions $f:[0,1]\to \IH$ where $Z_\tau(\gamma_e)$ has phase $\psi+\frac{\pi}{2} \mod \pi$ along each edge with charge $\gamma_e$.  
The tree is given as a nested list of initial objects of the form $k Ch[p/q][n]$ 
with $k,p,q,n$ integers.}
\defn{TreeToRaysPlot}{\vardef{Tree},\vardef{psi},\vardef{plotoptions}}
{Plots the rays produced by \fun{TreeToRays[Tree,psi]} with the given plot options.}

\section{Higgs branch formula}
In addition, the package includes some routines from the Mathematica 
package {\tt CoulombHiggs.m} \cite{CoulombHiggs}, mainly for the 
purpose of evaluating the indices $K_m(p,q)$ of the Kronecker quiver.
The names of the routines are prefaced by {\tt P2} to avoid clash.



\defn{P2HiggsBranchFormula}{\vardef{Mat},\vardef{Cvec},\vardef{Nvec} }
	                                   {standalone routine which computes the Poincar\'e-Laurent
                   polynomial  of a quiver with DSZ products $\alpha_{ij}=\var{Mat}[[i,j]]$, 
                   dimension vector $N_i=\var{Nvec}[[i]]$, 
                   FI parameters $\zeta_i=\var{Cvec}[[i]]$, using Reineke's formula. 
                   It is assumed, but not checked, that the quiver has no oriented 
                   loop;}


\defn{P2StackInvariant}{\vardef{Mat},\vardef{Cvec},\vardef{Nvec},\vardef{y} }{gives 
                 the stack 
                  invariant $G_{\rm Higgs}(\gamma,\zeta,y)$
                  of a quiver with DSZ matrix $\alpha_{ij}=\var{Mat}[[i,j]]$, possibly rescaled 
                  by an overall factor of 
                   \var{{\tt \$QuiverMultiplier}}, 
                  FI parameters $\zeta_i=\var{Cvec}[[i]]$,  dimension vector $N_i=\var{Nvec}[[i]]$, 		using Reineke's formula; the answer is written in terms of unevaluated
		$q$-deformed factorials \var{P2QFact[n,y]};}

\defn{P2BinarySplits}{\vardef{Nvec}}{gives the list of dimension vectors $\gamma_L$ less than $\gamma$, quotiented by the equivalence relation $\gamma_L \to \gamma-\gamma_L$.}

\defn{P2OmToOmb}{\vardef{f}}{expresses any $\Omega(\gamma,y)$ in $f$
 in terms of $\bar\Omega(\gamma,y)$'s;}
 
 \defn{P2OmbToHiggsG}{\vardef{Cvec},\vardef{f} }{expresses  any 
 $\bar\Omega(\gamma,y)$ in $\var{f}$
in terms of the (unevaluated) stack invariants $\gR(\gamma,y)$ .}

 \defn{P2EvalHiggsG}{\vardef{Cvec},\vardef{f} }{evaluates any 
$\gR(\gamma,y)$  in $\var{f}$
using \fun{P2StackInvariant}}

\defn{P2ListAllPartitions}{\vardef{gam}}{returns the list of unordered 
                   partitions  of the positive integer vector $\gamma$ as a sum of positive, non-zero integer vectors $\alpha_i$; 
                   }

\defn{P2QDeformedFactorial}{\vardef{n},\newdefy }{gives the $q$-deformed factorial $[n,y]!$}

\defn{P2EvalQFact}{\vardef{f}}{evaluates any \var{P2QFact[n,y]} appearing in \var{f}}



\bibliography{combined}
\bibliographystyle{utphys}


\end{document}

